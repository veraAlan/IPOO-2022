\documentclass{article}
\usepackage[paperheight=297mm, paperwidth=210mm, top=15mm, bottom=20mm, left=25mm, right=25mm]{geometry}
\usepackage{hyperref}
\title{Documento de Ejercicios del \textit{TP1} de \textit{IPOO-2022}.}
\author{\textbf{\textit{Alan Cristian Gaston Vera}}}
\date{}

\begin{document}
\maketitle
\section*{Ejercicio 1.}

\textbf{a. Objeto: }\\
\indent \indent \indent \indent En general los objetos son una forma ordenada de agrupar datos y operaciones a realizar sobre ellos.
\\
\textbf{b. Clase:}\\
\indent \indent \indent \indent Definición del tipo, donde serán indicados los atributos y metodos que van a tener todas las variables que sean de ese tipo.
\\
\textbf{c. Método:}\\
\indent \indent \indent \indent Define cómo se va a comportar el objeto. Ej: Una persona puede hablar, caminar, etc.
\\
\textbf{d. Atibuto:}\\
\indent \indent \indent \indent Define el estado del objeto. Ej: Una persona tiene un nombre, una edad, una altura, etc.

\section*{Ejercicio 2.}

\quad Ambos nombres \textit{(\textbf{variable instancia} e \textbf{instancia de una clase})} hacen
referencia a una variable que ha sido creada como tipo de objeto de una clase.
Esto quiere decir que una variable instancia es de un tipo de objeto,
pero cabe destacar que esta puede ser diferente a otra variable que es instancia del mismo objeto. \par
\textbf{Ej: }\\
\indent var1 = new Clase1();\\ \indent var2 = new Clase1();\\
\indent Las variables \textit{\textbf{var1}} y \textit{\textbf{var2}} son instancias de la clase \textbf{Clase1}. Tienen atributos y métodos similares pero estas variables no son iguales.

\section*{Ejercicio 3.}

\textbf{(C) = Clase\\(M) = Método\\(A) = Atributo}\\
\begin{tabular}{llll}
    mouse (C)                 & reloj (C)           & persona (C)         & inalambrico (A)       \\
    televisor (C)             & mover (M)           & darHoraActual (M)   & ponerHora (M)         \\
    darCanalActual (M)        & fechaNacimiento (A) & irCanalTv (M)       & documento (C)         \\
    configurarDespertador (M) & encender (M)        & fechaNacimiento (A) & darPosicionActual (M)
\end{tabular}

\section*{Ejercicio 4.}

\begin{enumerate}
    \item Reloj: \\\textbf{Atributos:}
          \\\textit{hora, minutos, segundos}
          \\\textbf{Métodos:}
          \\\textit{darHoraActual, ponerHora, incrementarHora, incrementarMinutos, incrementarSegundos}
    \item Taza: \\\textbf{Atributos:}
          \\\textit{capacidad, color}
          \\\textbf{Métodos:}
          \\\textit{darCapacidad, ponerCapacidad}
          \newpage
    \item Billetera: \\\textbf{Atributos:}
          \\\textit{dineros, tarjetas}
          \\\textbf{Métodos:}
          \\\textit{darDinero, ponerDinero}
    \item Monitor: \\\textbf{Atributos:}
          \\\textit{resolucion, marca, ppi, led}
          \\\textbf{Métodos:}
          \\\textit{darResolucion, prenderLED, apagarLED}
    \item Celular: \\\textbf{Atributos:}
          \\\textit{marca, modelo, color, sistemaOperativo}
          \\\textbf{Métodos:}
          \\\textit{encender, darMarca, ponerMarca, darModelo, darColor, darSistemaOperativo}
\end{enumerate}

\section*{Ejercicio 5.}

\indent Archivo Persona.php encontrado en carpeta TP1. \par

\indent \textbf{Clase Persona:}
\begin{itemize}
    \item \textbf{Atributos:}\\
          \textit{nombre, edad, altura, peso, sexo}
    \item \textbf{Métodos:}\\
          \textit{Setters y getters de cada atributo, construct, desctruct, toString, crecerAltura y cumplirAnios}
\end{itemize}

\section*{Ejercicio 6.}

Los métodos que deben estar en todos las clases de objetos son:
\begin{itemize}
    \item \textbf{Constructor: \textit{"\_\_construct()\{\}}"}\\\textit{Se ejecuta cuando se crea un objeto de la clase.}
    \item \textbf{Destructor: \textit{"\_\_destruct()\{\}}"}\\\textit{Se ejecuta cuando se destruye un objeto de la clase.}
\end{itemize}

\section*{Ejercicio 7.}

\indent El método \textit{"\textbf{\_\_toString()}"} sirve para visualizar los valores
de las variables instancias de un objeto. Este método retorna un string
(Cadena de caracteres) con la información a visualizar y es invocado cada
vez que se llama a la función \textit{echo}.

\section*{Ejercicio 8.}

\indent El método que se ejecuta cuando se crea una instancia de una clase es el
método \textit{\textbf{\_\_construct()}}. Este es denominado el método constructor
el cual su fúncionalidad es inicializar la clase.

\section*{Ejercicio 9.}

Cuando se utiliza \textit{\textbf{\$this}} cómo parametro de un método se referencia
a la instancia actual.

\section*{Ejercicio 10.}

Clases y métodos a implementar:
\begin{itemize}
    \item a. \textbf{Calculadora:} \\\textit{suma, resta, multiplicacion, division}
    \item b. \textbf{Reloj:} \\\textit{puesta\_a\_cero, incremento, puesta\_a\_tiempo}
    \item c. \textbf{Fecha:} \\\textit{incrementar\_un\_dia, fecha\_abreviada, fecha\_extendida}
    \item d. \textbf{Login:} \\\textit{validar\_contrasenia, cambiar\_contrasenia, recordar}
\end{itemize}

\section*{Ejercicio 11.}

\textbf{Clase:} \textit{Cuadrado} \\
\textbf{Atributos:} \$v1, \$v2, \$v3, \$v4 \\
\textbf{Métodos:}
\begin{itemize}
    \item \textbf{\_\_constructor():} \\ Recibe cómo parametros las coordenadas de los puntos.
    \item \textbf{area():} \\ Retorna el área del cuadrado.
    \item \textbf{desplazar(\$d):} \\ Recibe por parámetro un punto y desplaza el cuadrado (Centrandose en el vertice inferior izquierdo).
    \item \textbf{aumentarTamaño(\$t)} \\  Recibe por parametro el tamaño a aumentar el cuadrado.
    \item \textbf{\_\_toString():} \\ Retorna la información de los atributos de la clase.
    \item \textbf{\_\_destructor():}  \\ Llamado en cualquier tipo de finalización de la instancia.
\end{itemize}
    
Crear una instancia de la clase Cuadrado y probar sus funciones en un script testCuadrado.php \par

\section*{Ejercicio 12.}

\textbf{Clase:} \textit{Linea} \\
\textbf{Atributos:} \$p1, \$p2 \\
\textbf{Métodos:}


\end{document}